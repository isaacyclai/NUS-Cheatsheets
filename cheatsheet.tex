\documentclass[11pt]{article}

\usepackage{amsfonts,latexsym,amsthm,amssymb,amsmath,amscd,euscript}
\usepackage{enumerate}
\usepackage{eqparbox}
\usepackage{framed}
\usepackage{fullpage}
\usepackage{hyperref}
    \hypersetup{colorlinks=true,citecolor=blue,urlcolor =black,linkbordercolor={1 0 0}}
\usepackage{blindtext}
\usepackage{bm}
\usepackage{asymptote}
\usepackage{mathtools}

\usepackage[dvipsnames]{xcolor}
\usepackage[framemethod=TikZ]{mdframed}
\usepackage[most]{tcolorbox}

\usepackage{minted}

\theoremstyle{remark}
\newtheorem*{remark}{Remark}
\newtheorem*{notation}{Notation}
\newtheorem*{note}{Note}

\newcommand{\BR}{\mathbb R}
\newcommand{\BC}{\mathbb C}
\newcommand{\BF}{\mathbb F}
\newcommand{\BQ}{\mathbb Q}
\newcommand{\BZ}{\mathbb Z}
\newcommand{\BN}{\mathbb N}

\newcommand{\mb}[1]{\mathbf #1}

\newcommand{\floor}[1]{\left\lfloor #1 \right\rfloor}
\newcommand{\ceiling}[1]{\left\lceil #1 \right\rceil}
\newcommand{\abs}[1]{\left\lvert #1 \right\rvert}
\newcommand{\norm}[1]{\left\lVert #1 \right\rVert}
\newcommand{\innerproduct}[2]{\left\langle #1, #2 \right\rangle}

\newcommand{\ddx}{\frac{d}{dx}}
\newcommand{\curl}{\text{curl }}
\newcommand{\dvg}{\text{div }}

\newcommand{\adj}{\text{adj }}

\DeclareMathOperator*{\argmax}{arg\,max}
\DeclareMathOperator*{\argmin}{arg\,min}

\title{CS3230 midterm reference}
\date{\today}
\author{Isaac Lai}

\begin{document}
\maketitle
\section{Asymptotic facts}
\begin{align*}
    e^x&\geq 1+x\\
    a^{\log_b c}&=c^{\log_b a}\\
    n!&=\sqrt{2\pi n}\left(\frac{n}{3}\right)^n\left(1+\Theta\left(\frac{1}{n}\right)\right)\text{
    (Stirling's approximation)}\\
    \log(n!)&=\Theta(n\log n)\\
    \sum_{k=0}^n ar^k&= \frac{a(r^{n+1}-1)}{r-1}\text{ (Geometric series)}\\
    \sum_{k=1}^n \frac{1}{k}&=\ln n+O(1)\text{ (Harmonic series)}\\
    \lim_{x\to\infty}\frac{f(x)}{g(x)}&=\lim_{x\to\infty}\frac{f'(x)}{g'(x)}\text{ (L'Hopital's Rule)}\\
\end{align*}
\section{Asymptotic analysis}
\begin{itemize}
    \item $f(n)=O(g(n))$ if  $\exists c>0,n_0>0$ such that $\forall n\geq n_0, 0\leq f(n)\leq cg(n)$
    \item  $f(n)=\Omega(g(n))$ if  $\exists c>0,n_0>0$ such that $\forall n\geq n_0, 0\leq cg(n)\leq f(n)$
    \item  $f(n)=\Theta(g(n))$ if $\exists c_1,c_2>0,n_0>0$ such that $\forall n\geq n_0, 0\leq c_1g(n)\leq
        f(n)\leq c_2g(n)$ i.e. $\Theta(g)=O(g)\cap\Omega(g)$ 
    \item $\lim\limits_{n\to\infty}\frac{f(n)}{g(n)}=0\implies f(n)=o(g(n))$ 
    \item $\lim\limits_{n\to\infty}\frac{f(n)}{g(n)}<\infty\implies f(n)=O(g(n))$    
    \item $0<\lim\limits_{n\to\infty}\frac{f(n)}{g(n)}<\infty\implies f(n)=\Theta(g(n))$
    \item $\lim\limits_{n\to\infty}\frac{f(n)}{g(n)}>0\implies f(n)=\Omega(g(n))$  
    \item $\lim\limits_{n\to\infty}\frac{f(n)}{g(n)}=\infty\implies f(n)=\omega(g(n))$   
\end{itemize}
\section{Recurrences}
General form: $T(n)=aT(\frac{n}{b})+f(n)$
\begin{itemize}
    \item Telescoping: express in form $\frac{T(n)}{g(n)}=\frac{T(\frac{n}{b})}{g(\frac{n}{b})}+h(n)$ 
    \item Recursion tree: draw tree, sum each node (can sum over level first then over height)
    \item \textbf{Master Theorem}: $a\geq 1,b>1,f$ asymptotically positive
        \begin{enumerate}
            \item $f(n)=O(n^{\log_b a-\epsilon})$ for some  $\epsilon>0$.  $T(n)=\Theta(n^{\log_b a})$
            \item  $f(n)=\Theta(n^{\log_b a}\log^k n)$ for some  $k\geq 0$.  $T(n)=\Theta(n^{\log_b
                a}\log^{k+1}n)$
            \item  $f(n)=\Omega(n^{\log_b a+\epsilon})$ for some  $\epsilon>0$ and satisfies
                \textbf{regularity condition}  $af(\frac{n}{b})\leq cf(n)$ for some $c<1$.
                $T(n)=\Theta(f(n))$
        \end{enumerate}
    \item Substitution: guess and check by induction. Induction hypothesis: $c_1n^k-\text{(lower order
        terms)}$
\end{itemize}
\section{Divide and Conquer}
Invariant: condition which is true at the start of every iteration. To show correctness check
\begin{itemize}
    \item Initialisation: invariant is true at iteration 1
    \item Maintenance: if invariant is true for iteration $n$, it remains true for iteration $n+1$ 
    \item Termination: when the algorithm ends, the invariant helps the proof of correctness
\end{itemize}
\section{Randomisation}
\begin{itemize}
    \item Las Vegas algorithms: (1) output always correct (2) running time depends on random bits, with
        small probability that it may be large (3) expected running time is bounded by given time-bound
        function
    \item Monte Carlo algorithms: (1) answer may be incorrect with small probability (2) running time is always bounded by given time-bound function
\end{itemize}
\section{Dynamic Programming}
\begin{itemize}
    \item Optimal substructure: optimal solution contains optimal solutoins to subproblems
    \item Overlapping subproblems: recursive solution contains a small number of distinct subproblems
        repeated many times
    \item Top-down saves computation of unnecessary subproblems but can suffer from
        overhead of recursive calls, bottom-up is the opposite
\end{itemize}
\end{document}
